% 2024-09-30

% spacing (horizontal) in LaTeX
% https://tex.stackexchange.com/q/74353
% https://www.overleaf.com/learn/latex/Spacing_in_math_mode

\documentclass{article}
\usepackage[a4paper,margin=1.00cm]{geometry}
\usepackage[table]{xcolor}
% the 'table' option ('xcolor' package) loads the 'colortbl' package,
% which requires the 'color' and 'array' packages
%\usepackage{array}   % required for '!{}'

\begin{document}
\pagestyle{empty}
\centering

\begin{tabular}{lllll}
  \multicolumn{5}{c}{\textbf{LaTeX spacing}} \\ \\
  H\!H & \verb*|H\!H| & $H\!H$ & \verb*|$H\!H$|
  & -3mu = -3/18 of \verb*|\quad| or \verb*|\negthinspace| \\
  H H  & \verb*|H H|  & $H H$  & \verb*|$H H$|
  & \textbf{space} or \verb*|{ }| \verb*|\space| \\
  H\,H & \verb*|H\,H| & $H\,H$ & \verb*|$H\,H$|
  & 3mu = 3/18 of \verb*|\quad| or \verb*|\thinspace| \\
  H\:H & \verb*|H\:H| & $H\:H$ & \verb*|$H\:H$|
  & 4mu = 4/18 of \verb*|\quad| or \verb*|\medspace| \\
  H\;H & \verb*|H\;H| & $H\;H$ & \verb*|$H\;H$|
  & 5mu = 5/18 of \verb*|\quad| or \verb*|\thickspace| \\
  H\ H & \verb*|H\ H| & $H\ H$ & \verb*|$H\ H$|
  & control space (normal) \\
  H H  & \verb*|H H|  & $H H$  & \verb*|$H H$|
  & \textbf{space} or \verb*|{ }| \verb*|\space| \\
  H  H & \verb*|H  H| & $H  H$ & \verb*|$H  H$|
  & spaces \\
  H{\quad}H & \verb*|H{\quad}H| & $H{\quad}H$
  & \verb*|$H{\quad}H$| & 18 mu = \verb*|\quad|
\end{tabular}
\vspace{2em}

\begin{tabular}{lll}
  \multicolumn{3}{c}{\textbf{Reaction LaTeX typeset}} \\ \\
  ${}^{4}\!H\!e\, p\, \!\to {}^{3}\!H\!e\, p\, p\, \pi^{-} \pi^{0}$ &
  \verb*|${}^{4}\!H\!e\, p\, \!\to {}^{3}\!H\!e\, p\, p\, \pi^{-} \pi^{0}$| & {\color{red} advanced} \\
  ${}^{4}He p \to {}^{3}He p p \pi^{-} \pi^{0}$ &
  \verb*|${}^{4}He p \to {}^{3}He p p \pi^{-} \pi^{0}$| & simple \\
\end{tabular}
\vspace{2em}

\begin{tabular}[t]{lll}
  % https://www.overleaf.com/learn/latex/Errors/Illegal_unit_of_measure_(pt_inserted)
  % If a linebreak command '\\' is ever followed by square brackets '[]' it will generate error
  \multicolumn{3}{c}{\textbf{Interval LaTeX typeset}} \\ \\
  & [-123,-123]       & \verb*|[-123,-123]| \\
  & [-123, -123]      & \verb*|[-123, -123]| \\
  & [ -123, -123]     & \verb*|[ -123, -123]| \\
  & [\,-123,-123\,]   & \verb*|[\,-123,-123\,]| \\
  & [\,-123,\,-123\,] & {\color{red} \verb*|[\,-123,\,-123\,]|} \\
  & [\,-123, -123\,]  & \verb*|[\,-123, -123\,]| \\
\end{tabular}
\quad
\begin{tabular}[t]{lll}
  \multicolumn{3}{c}{\textbf{Interval LaTeX indent of numbers}} \\ \\
  & [\, 1,\, 2\,]                     & \verb*|[\, 1,\, 2\,]| \quad (wrong indent) \\
  & [\,11,\,22\,]                     & \verb*|[\,11,\,22\,]| \\
  & [\,\phantom{0}1,\,\phantom{0}2\,] & \verb*|[\,\phantom{0}1,\,\phantom{0}2\,]| \\
  & [\,11,\,\phantom{0}2\,]           & \verb*|[\,11,\,\phantom{0}2\,]| \\
  & [\,\phantom{0}1,\,22\,]           & \verb*|[\,\phantom{0}1,\,22\,]| \\
\end{tabular}
\vspace{2em}

% https://en.wikibooks.org/wiki/LaTeX/Tables#@_and_!_expressions
% '@{bla}' suppresses inter-column space and inserts 'bla' instead (plain LaTeX)
% '!{bla}' keep initial space and inserts 'bla' (requires the 'array' package)
\begin{tabular}{|l|@{}l|l@{}|@{}l@{}|!{xx}l!{xx}|@{xx}l@{xx}|@{\hspace{2em}}l@{\hspace{2em}}|}
  \hline \multicolumn{7}{|c|}{\textbf{LaTeX tabular spacing}} \\ \hline
  \multicolumn{7}{|@{leftA}c@{rightA}|}{\textbf{LaTeX tabular spacing}} \\ \hline
  \multicolumn{7}{|!{leftB}c!{rightB}|}{\textbf{LaTeX tabular spacing}} \\ \hline
  column 1 & column 2 & column 3 & column 4 & column 5 & column 6 & column 7 \\ \hline
  column 1 & column 2 & column 3 & column 4 & column 5 & column 6 & column 7 \\ \hline
\end{tabular}
\vspace{2em}

% https://tex.stackexchange.com/q/16604
% https://tex.stackexchange.com/q/180357

%% \newcolumntype{H}{>{\setbox0=\hbox\bgroup}c<{\egroup}@{}} % OK1, hidden column
%% \newcolumntype{H}{>{\setbox0=\hbox\bgroup\cellcolor{green}}c<{\egroup}@{}}
%%
%% \newcolumntype{H}{>{\lrbox0}c<{\endlrbox}@{}} % % OK2, hidden column
%% \newcolumntype{H}{>{\lrbox0\cellcolor{blue}}c<{\endlrbox}@{}}

% columns align with '@{}' (inter-column space)
\newcolumntype{H}{@{}>{\setbox0=\hbox\bgroup}c<{\egroup}@{}}
\begin{tabular}{|c|Hc|}
  \hline \multicolumn{3}{|c|}{he4p@13.5, ver: 2024-08-08, 1} \\ \hline
  \multicolumn{1}{|c|}{Id} &
  \multicolumn{1}{@{}c@{}}{} &
  \multicolumn{1}{c|}{Reaction channel column} \\ \hline
  1234 & hidden text & more text \\ \hline
  4321 & hidden hiddent text & more more more text \\ \hline
\end{tabular}

\begin{tabular}{|c|c|}
  \hline \multicolumn{2}{|c|}{he4p@13.5, ver: 2024-08-08, 2} \\ \hline
  \multicolumn{1}{|c|}{Id} &
  \multicolumn{1}{c|}{Reaction channel column} \\ \hline
  1234 & more text \\ \hline
  4321 & more more more text \\ \hline
\end{tabular}

\newif\ifHIDDENCOLUMN
\newcommand{\multicolumnHidden}{
  \ifHIDDENCOLUMN \multicolumn{1}{@{}c@{}}{}
  \else           \multicolumn{1}{c|}{Hidden} \fi }

\HIDDENCOLUMNtrue
\ifHIDDENCOLUMN \newcolumntype{Z}{@{}>{\setbox0=\hbox\bgroup}c<{\egroup}@{}}
\else           \newcolumntype{Z}{c|} \fi
%%\else         \newcolumntype{Z}{>{}c<{}|} \fi   % the same

%% \newcolumntype{Z}{
%%   >{{\ifHIDDENCOLUMN \setbox0=\hbox\bgroup \fi}}c
%%   <{{\ifHIDDENCOLUMN \egroup \fi}}
%%   {\ifHIDDENCOLUMN @{} \else | \fi}}   % error with this condition

\begin{tabular}{|c|Zc|}   % identical tables
  \hline \multicolumn{3}{|c|}{he4p@13.5, ver: 2024-08-08, 3} \\ \hline
  \multicolumn{1}{|c|}{Id} &
  \multicolumnHidden &
  \multicolumn{1}{c|}{Reaction channel column} \\ \hline
  1234 & hidden text & more text \\ \hline
  4321 & hidden hiddent text & more more more text \\ \hline
\end{tabular}

\HIDDENCOLUMNfalse
\ifHIDDENCOLUMN \newcolumntype{Z}{@{}>{\setbox0=\hbox\bgroup}c<{\egroup}@{}}
\else           \newcolumntype{Z}{c|} \fi
%%\else         \newcolumntype{Z}{>{}c<{}|} \fi   % the same

\begin{tabular}{|c|Zc|}   % identical tables
  \hline \multicolumn{3}{|c|}{he4p@13.5, ver: 2024-08-08, 4} \\ \hline
  \multicolumn{1}{|c|}{Id} &
  \multicolumnHidden &
  \multicolumn{1}{c|}{Reaction channel column} \\ \hline
  1234 & hidden text & more text \\ \hline
  4321 & hidden hiddent text & more more more text \\ \hline
\end{tabular}

\end{document}
