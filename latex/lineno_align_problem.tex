% 2021-03-17

\documentclass[twocolumn]{article}

\usepackage[a4paper, margin=25mm]{geometry}
\usepackage[T1]{fontenc}
\usepackage{amsmath}

\usepackage[switch]{lineno}
%\renewcommand\linenumberfont{\normalfont\small\sffamily}
\linenumbers

%%%%%%%%%%%%%%%%%%%%%%%%%%%%%%%%%%%%%%%%
%%% https://tex.stackexchange.com/questions/43648/why-doesnt-lineno-number-a-paragraph-when-it-is-followed-by-an-align-equation
%%% https://texblog.org/2012/02/08/adding-line-numbers-to-documents/
%%% korektne align pre rozne enviroment
% \newcommand*\patchAmsMathEnvironmentForLineno[1]{
%   \expandafter\let\csname old#1\expandafter\endcsname\csname #1\endcsname
%   \expandafter\let\csname oldend#1\expandafter\endcsname\csname end#1\endcsname
%   \renewenvironment{#1}
%   {\linenomath\csname old#1\endcsname}
%   {\csname oldend#1\endcsname\endlinenomath}}
% \newcommand*\patchBothAmsMathEnvironmentsForLineno[1]{
%   \patchAmsMathEnvironmentForLineno{#1}
%   \patchAmsMathEnvironmentForLineno{#1*}}
% \AtBeginDocument{
%   \patchBothAmsMathEnvironmentsForLineno{equation}
%   \patchBothAmsMathEnvironmentsForLineno{align}
%   \patchBothAmsMathEnvironmentsForLineno{flalign}
%   \patchBothAmsMathEnvironmentsForLineno{alignat}
%   \patchBothAmsMathEnvironmentsForLineno{gather}
%   \patchBothAmsMathEnvironmentsForLineno{multline}
% }
%%%%%%%%%%%%%%%%%%%%%%%%%%%%%%%%%%%%%%%%

\thispagestyle{empty}
\begin{document}

Specificky problem pri pouziti balika \texttt{lineno}. Tento problem vznikol pri
pisani clanku s pouzitim SVJour3 LaTeX document class for Springer journals
(EPJA). Tento class pri pisani matematickych formul pouziva "nejaky align" aj v
prostredi \texttt{equation}.

Pre ilustraciu problemu pouzijeme prostredie \texttt{align}, pricom je
podstatne, ci pred prostredim (vzorcom) je \textbf{prazdny riadok}.
\begin{align}
  1+1=2 \hspace{0.5cm}\text{bez prazdneho riadku }\texttt{align}
\end{align}
a pokracujeme pisanim dalsieho super zloziteho matematickeho vzorca
\begin{align}
  2+2=4 \hspace{0.5cm}\text{bez prazdneho riadku }\texttt{align}
\end{align}
koniec ilustracie problemu.

\vspace{0.5cm}
\hrule
\vspace{0.5cm}
Specificky problem pri pouziti balika \texttt{lineno}. Tento problem vznikol pri
pisani clanku s pouzitim SVJour3 LaTeX document class for Springer journals
(EPJA). Tento class pri pisani matematickych formul pouziva "nejaky align" aj v
prostredi \texttt{equation}.

Pre ilustraciu problemu pouzijeme prostredie \texttt{align}, pricom je
podstatne, ci pred prostredim (vzorcom) je \textbf{prazdny riadok}.
\begin{align}
  1+1=2 \hspace{0.5cm}\text{bez prazdneho riadku }\texttt{align}
\end{align}
a pokracujeme pisanim dalsieho super zloziteho matematickeho vzorca

\begin{align}
  2+2=4 \hspace{0.5cm}\text{prazdny riadok pred }\texttt{align}
\end{align}
koniec ilustracie problemu.

\vspace{0.5cm}
\hrule
\vspace{0.5cm}
Specificky problem pri pouziti balika \texttt{lineno}. Tento problem vznikol pri
pisani clanku s pouzitim SVJour3 LaTeX document class for Springer journals
(EPJA). Tento class pri pisani matematickych formul pouziva "nejaky align" aj v
prostredi \texttt{equation}.

Pre ilustraciu problemu pouzijeme prostredie \texttt{align}, pricom je
podstatne, ci pred prostredim (vzorcom) je \textbf{prazdny riadok}.

\begin{align}
  1+1=2 \hspace{0.5cm}\text{prazdny riadok pred }\texttt{align}
\end{align}
a pokracujeme pisanim dalsieho super zloziteho matematickeho vzorca

\begin{align}
  2+2=4 \hspace{0.5cm}\text{prazdny riadok pred }\texttt{align}
\end{align}
koniec ilustracie problemu.

\newpage
Specificky problem pri pouziti balika \texttt{lineno}. Tento problem vznikol pri
pisani clanku s pouzitim SVJour3 LaTeX document class for Springer journals
(EPJA). Tento class pri pisani matematickych formul pouziva "nejaky align" aj v
prostredi \texttt{equation}.

V tomto priklade pre prostredie \texttt{equation} nevznika tento problem,
ale prejavuje sa s pouzitim SVJour3 LaTeX document class.
\begin{equation}
  1+1=2 \hspace{0.5cm}\text{bez prazdneho riadku }\texttt{equation}
\end{equation}
a pokracujeme pisanim dalsieho super zloziteho matematickeho vzorca
\begin{equation}
  2+2=4 \hspace{0.5cm}\text{bez prazdneho riadku }\texttt{equation}
\end{equation}
koniec ilustracie problemu.

\vspace{0.5cm}
\hrule
\vspace{0.5cm}
\textbf{Odkomentovanim preambuly v zdrojovom kode \LaTeX, bude balik
  \texttt{lineno} korektne cislovat riadky nezavisle od pouziteho prostredia.}

\end{document}
