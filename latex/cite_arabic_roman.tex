% 2009-01-26 original
% 2019-05-13 cosmetic update

\documentclass[12pt,oneside,slovak]{book}

\usepackage[utf8]{inputenc}
\usepackage[T1]{fontenc}
\usepackage[slovak]{babel}
\usepackage{lmodern}

% !!! mucha: begin
\usepackage{cite}   % potrebny package

\makeatletter
\newcommand{\citer}[0]{\renewcommand\citeform[1]   % cite roman
  {\romannumeral ##1}}
\newcommand{\citeR}[0]{\renewcommand\citeform[1]   % cite Roman
  {\expandafter\@slowromancap\romannumeral ##1@}}
\newcommand{\cited}[0]{\renewcommand\citeform[1]   % cite default (arabic)
  {##1}}
%\newcommand{\citea}[0]{\renewcommand\citeform[1]   % cite arabic (manual)
%  {\@arabic ##1}}
\makeatother

% ak nechcem pouzivat "nove prikazy", mozem menit pocitadlo
% aj manulane, stale pred zmenenim cislovania pred prikazom '\cite'
% napisem jeden z tychto prikazov (Roman, roman, default):
% \renewcommand\citeform[1]{\expandafter\@slowromancap\romannumeral #1@}
% \renewcommand\citeform[1]{\romannumeral #1}
% \renewcommand\citeform[1]{##1}
% !!! mucha: end

\setcounter{totalnumber}{50}
\setcounter{topnumber}{50}
\setcounter{bottomnumber}{50}


\begin{document}

{ % curly braces must be
  \renewcommand{\chaptername}[2]{Úvod}
  \chapter{}

  So štúdiom produkcie mezónov v hadrónovo-hadrónových interakciách v blízkosti
  kinematického prahu sa začalo v pätdesiatych rokoch, ked sa objavili
  urýchlovače, ktoré poskytli zväzky protónov s dostatočne vysokou
  energiou. Silná vzájomná prepojenost medzi rozvojom urýchlovačovej fyziky,
  pokrokom v oblasti detekčných metod a teoretickou interpretáciou viedla k
  burlivému rozmachu tejto oblasti fyziky. Už experimenty vykonané pomocou
  bublinových komôr naznačili niektoré typické črty týchto procesov, ktoré boli
  potom konfrontované s výsledkami ovela komplexnejších štúdií v nukleónových
  zväzkoch na urýchlovačoch TRIUMF, LAMPF, PSI, LEAR a SATURNE. Nástup nových
  urýchlovačov s chladenými zväzkami ako IUCF, CELSIUS a COSY otvoril novú éru
  precíznych experimentov a znamenal dôležitý krok v skúmaní procesov produkcie
  mezónov v blízkosti kinematického prahu. Produkcia mezónov bola práve jedným z
  nosných programov troch vyššie uvedených centier.
} % curly braces must be

\setcounter{chapter}{0}
\chapter{Popis experimentálnej apartúry kolaborácie GEM}

Experimentálna aparatúra GEM bola ožarovaná zväzkami synchrotrónu Cooler
Synchrotron (COSY), Forschungszentrum (FZ) Jülich a je realizovaná ako
kombinácia dvoch typov detektorov.
\begin{itemize}
\item
  Centrálny detektor s veľkou uhlovou akceptanciou, umiestnený pri terčíku. Je
  to jeden z detektorov buď germániová stena (Germanium Wall - GeWall) alebo
  ENSTAR.
\item
  Detektor s malou uhlovou akceptanciou magnetický spektrometer (spektrograf)
  Big Karl v súčinnosti s mnohovláknovými driftovými komorami a scintilátorovými
  hodoskopmi.
\end{itemize}

\section{Detektor ENSTAR}

Detektor ENSTAR bol navrhnutý v kombinácii s magnetickým spektrometrom Big Karl
na štúdium možnej produkcie $\eta$-mezónových jadier v reakciách typu:
$p + (^{Z}X_{A}) \rightarrow  ^{3}He + (^{Z-1}X_{A-2})\eta$, pričom rýchle jadrá
$^{3}He$ vylietavajúce pod malými uhlami ($ < 6^{\circ}$) sa detegujú
spektrometrom. Produkty rozpadu pomalého $N^{*}$ - protón a $\pi$  mezón budú
registrované detektorom ENSTAR, ktorý pozostáva z troch vrstiev scintilačných
detektorov, vytvárajúcich koncentrický válec. Tenká vnútorná vrstva hrá úlohu
detektora $\Delta$ E a dve hrubé vonkajšie vrstvy merajú depozitovanú energiu.
Detektor je členený tak, aby umožnil stanovenie azimutálneho a polárneho uhla
častíc. Podrobný popis detektora sa nachádza  v priloženej publikácii
% !!! mucha: odteraz sa bude stale cislovat Roman
\citeR \cite{EnstarDetector}.

\section{Magnetický spektrometer Big Karl}

Magnetický spektrometer pozostáva z dvoch dipolóvych a zo štyroch quadrupólových
magnetov a slúži na fokusáciu častíc vylietavajúcich v úzkom kúželi okolo
$0^{\circ}$. Podľa veľkosti rigidity častice buď ju transportuje do štandardnej
fokálnej roviny alebo častica opustí spektrometer cez výstup prvého dipólu. Na
dvoch výstupoch sú umiestnené dva identické detekčné systémy, pozostávajúce z
dvoch mnohovláknových driftových komôr a zo scintilačného hodoskópu. Trajektória
častíc je popísaná transportnou maticou. V dobrom priblížení sa dá povedať, že:
horizontálna pozícia je daná hybnosťou častice, horizontálny uhol a vertikálna
pozícia sú určené horizontálnym a vertikálnym uhlom trajektórie v terčíku. Uhly
a súradnice vo výstupnych rovinách sa dajú určiť pomocou driftových komôr.
Scintilačné detektory merajú dobu preletu častíc. Podrobný popis magnetického
spektrometra spolu s ďalšími referenciami možno nájsť v priloženej publikácii
\cite{BigKarl}
% !!! mucha
a este \cite{QuarkMassDiffViaISBGeWall} a stale velkym Romanom a este dalsia
literatura \cite{PiEtaMixpdResultsBigKarl, BoundEtaJha}.

A teraz čo napíšeš
% !!! mucha: odteraz sa bude stale cislovat default (t.j. arabic)
\cited \cite{PDG2004} teraz bude stale \cite{Ueda} cislovat default, t.j. v
nasom pripade arabic \cite{Willisetal}.


{ % curly braces must be
  \renewcommand{\bibname}{Zoznam priloženej literatúry}
  % !!! mucha: begin
  \makeatletter
  \renewcommand{\@biblabel}[1]{[\Roman{enumiv}]\indent\indent}
  %\renewcommand{\@biblabel}[1]{[\roman{enumiv}]\indent}
  \makeatother
  % !!! mucha: end
  \begin{thebibliography}{VIII}  %for 2 digits

  \bibitem{EnstarDetector}
    M.G. Betigeri, ... J. Urbán, ... et al., A large acceptance scintillator
    detector with wavelength shifting fibre readout for search of $\eta-$
    nucleus bound states \\
    Nucl. Instr. and Meth., A 578 (2007) 198-206.

  \bibitem {BigKarl}
    J. Bojowald, ... J. Urbán, ... et al., Magnetic spectrometer Big Karl for
    studies of meson production reactions \\
    Nucl. Instr. and Meth., A 487 (2002) 314-322.

  \bibitem{SimMeaspdGeWall}
    S. Abdel-Samad, ... J. Urbán, ... et al., Simultaneous Measurements of $p+d
    \rightarrow ^{3}He + \pi^{0}$ and $p+d \rightarrow ^{3}H + \pi^{+}$
    Reactions with the GEM detector \\
    Physica Scripta T104 (2003) 88-90.

  \bibitem{QuarkMassDiffViaISBGeWall}
    H. Machner, ... J. Urbán, ... et al., Search for Up-Down Quark Mass
    Difference via Isospin Symmetry Breaking \\
    Progress in Particle and Nuclear Physics 50 (2003) 605-614.

  \bibitem{PiEtaMixpdResultsBigKarl}
    M. Abdel-Bary, ... J. Urbán, ... et al.,
    $\pi^{0}-\eta$ meson mixing in $p+d \rightarrow ^{3}H\pi^{+}/^{3}He\pi^{0}$
    reactions \\
    Phys. Rev. C 68 (2003) 021603(R)

  \bibitem{EtaMass}
    S. Abdel-Bary, ... J. Urbán, ... et al., A precision determination of the
    mass of the $\eta$ meson\\
    Phys. Lett. B 619 (2005) 281-287.

  \bibitem{BoundEtaJha}
    V. Jha,  ... J. Urbán, ... et al., Search for $\eta$-Nucleus Bound State at
    Cosy \\
    Int. Journal of Mod. Phys. A 22 (2007) 596-599.

  \end{thebibliography}
} % curly braces must be


\renewcommand{\bibname}{Zoznam použitej literatúry}

\begin{thebibliography}{99}  %for 2 digits

\bibitem {PDG2004}
  S. Eidelman et al., Phys. Lett. B 592 (2004) 1.

\bibitem{LagetLecolley}
  J.M. Laget a J.F Lecolley, Phys. Lett. B 194 (1987) 177.

\bibitem{Ueda}
  T. Ueda,  Nucl. Phys. A 505 (1989) 610.

\bibitem{MillerNefkensSlaus}
  G.A. Miller, B.M.K. Nefkens a I. \v{S}laus, Phys. Rep. 194 (1999) 1.

\bibitem{MagieraMachner}
  A. Magiera a H. Machner, Nucl. Phys. A 674 (2000) 515.

\bibitem{LiuSukromna}
  L.C. Liu, Súkromná informácia \footnote{Výsledky pochádzajú zo súkromnej
    komunikácie medzi L.C. Liu a J. Lieb, ktorý mi ich podstúpil.}

\bibitem{PDG1998}
  C. Caso et al., Eur. Journal of Physics, C 3 (1998) 109.

\bibitem{gem_eta_mass_proposal}
  http://www.fz-juelich.de/ikp/gem   Proposals, I/2000.

\bibitem{WycechGreenNiskanen}
  S. Wycech, A.M. Green, J.A. Niskanen, Phys. Rev. C 52 (1995) 544.

\bibitem{Rakytyanskyetal}
  S.A. Rakytyansky et al., Phys. Rev. 53 (1996) R2043.

\bibitem{FixArenhovel}
  A. Fix and H. Arenh\"{o}vel, Phys. Rev. C 66 (2002) 024002.

\bibitem{Willisetal}
  N. Willis et al., Phys. Lett. B 406 (1997) 14.

\end{thebibliography}

\end{document}
